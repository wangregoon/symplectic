\documentclass{paper}

\usepackage[margin=1in]{geometry} 
\usepackage{amsmath,amsthm,amssymb}
\usepackage{textcomp}
\usepackage[colorlinks,linkcolor=red,anchorcolor=blue,citecolor=green]{hyperref}

\newcommand{\N}{\mathbb{N}}
\newcommand{\Z}{\mathbb{Z}}
\newcommand{\R}[1]{\mathbb{R}^{#1}}
\newcommand{\tb}[1]{\textbf{#1}}
\newcommand{\ti}[1]{\textit{#1}}
\newcommand{\mca}[1]{\mathcal{#1}}
\newcommand{\map}[3]{{#1}:{#2}\rightarrow{#3}}

\newtheorem{problem}{Problem}
\newtheorem{thm}{Thm.}[section]
\newtheorem{prop}{Prop.}[section]
\newtheorem{lem}{Lem.}[section]
\newtheorem{cor}{Coro.}[section]
%\newtheorem{conj}[thm]{Conjecture}
\newtheorem{exer}{Exer}[section]

\theoremstyle{definition}
\newtheorem{definition}{Def.}[section]
%\newtheorem{example}{Example}[section] 

\theoremstyle{remark}
\newtheorem*{note}{Note}
\newtheorem*{remark}{\tb{Remark}}
\newtheorem*{claim}{\tb{Claim}}
\newenvironment{solution}
               {\let\oldqedsymbol=\qedsymbol
                \renewcommand{\qedsymbol}{$\blacktriangleleft$}
                \begin{proof}[\bfseries\upshape Solution]}
               {\end{proof}
                \renewcommand{\qedsymbol}{\oldqedsymbol}}
 
\begin{document}
 
\title{Symplectic Geometry}
\author{Regoon Wang, ChemE@UNSW \\ wang.regoon@gmail.com} 


\maketitle
\begin{abstract}
This is a note on Ana Cannas's \textit{Lectures on Symplectic Geometry}. However, it's not just a copy. It contains my
understanding, questions and solutions to homework. 
\end{abstract} 

\tableofcontents
\section{Notations}
In order to keep the text short, common used notations are introduced here.
\begin{enumerate}
\item[$\bullet$] V be an \textit{m}-dimensional vector space over $\mathbb{R}$. 
\item[$\bullet$] $\Omega:V\times V\rightarrow \mathbb{R}$ be a bilinear map.
\end{enumerate}
\section{Symplectic Forms}
\begin{definition}
The map $\Omega$  is \tb{skew-symmetric} if $\Omega(u,v)=-\Omega(v,u), \forall u,v\in V$.
\end{definition}
\begin{thm}[\tb{Standard Form for Skew-symmetric Bilinear Map}]
$\exists$ a basis $u_1,\cdots,u_k,e_1,\cdots,e_n,f_1,\cdots,f_n$ of V s.t. $\forall i,j \text{ and } v\in V$
\begin{equation*}
\Omega(u_i,v)=\Omega(e_i,e_j)=\Omega(f_i,f_j)=0,\quad\Omega(e_i,f_j)=\delta_{ij}
\end{equation*}
\end{thm}
\begin{remark*}
\begin{enumerate}
\item[1.] The basis is not unique, though it is traditionally also called a "canonical" basis.
\item[2.] In matrix notation with respect to such basis, we have 
\begin{equation*}
\Omega(u,v)=
\begin{bmatrix}
- & u & -
\end{bmatrix}
\begin{bmatrix}
0 & 0 & 0 \\
0 & 0 & I_n \\
0 &-I_n& 0
\end{bmatrix}
\begin{bmatrix}
| \\
v \\
|
\end{bmatrix}
\end{equation*}
and all normed linear independt enginvectors are basis. 
\end{enumerate}
\end{remark*}
\begin{proof}
This induction proof is a skew-symmetric version of the Gram-Schmidt process. Let $U:=\{u\in V\mid 
\Omega(u,v)=0,\forall v\in V\}$. Choose a basis $u_1,\cdots,u_k$, and choose a complemetary space W,
$V=U\oplus W$
\end{proof}
\subsection{Skew-Symmetric Bilinear Maps}
\subsection{Symplectic Vector Space}
\subsection{Symplectic Manifolds}
\subsection{Symplectomorphisms}
\subsection{Homework}
\section{Symplectic Form on the Cotangent Bundle}
\subsection{Cotangent Bundle}
\subsection{Tautological and Canonical Forms in Coordinates}
\subsection{Coordinate-Free Definitions}
\subsection{Naturality of the Tautological and Canonical Forms}
\subsection{Homework} 
\end{document}