\documentclass[4paper]{article}

\usepackage[margin=1in]{geometry} 
\usepackage{amsmath,amsthm,amssymb}
\usepackage{textcomp}
\usepackage[colorlinks,linkcolor=red,anchorcolor=blue,citecolor=green]{hyperref}

\newcommand{\N}{\mathbb{N}}
\newcommand{\Z}{\mathbb{Z}}
\newcommand{\R}[1]{\mathbb{R}^{#1}}
\newcommand{\tb}[1]{\textbf{#1}}
\newcommand{\ti}[1]{\textit{#1}}
\newcommand{\mca}[1]{\mathcal{#1}}
\newcommand{\map}[3]{{#1}:{#2}\rightarrow{#3}}

\newtheorem{problem}{Problem}
\newtheorem{thm}{Thm}
\newtheorem{prop}{Prop}
\newtheorem{lem}{Lem}
\newtheorem{cor}{Coro}
%\newtheorem{conj}[thm]{Conjecture}
\newtheorem{exer}{Exer}

\theoremstyle{definition}
\newtheorem{definition}{Def}
%\newtheorem{example}{Example}[section] 

\newtheorem{note*}{Note}
\theoremstyle{remark}
\newtheorem{remark*}{Remark}

\newenvironment{solution}
               {\let\oldqedsymbol=\qedsymbol
                \renewcommand{\qedsymbol}{$\blacktriangleleft$}
                \begin{proof}[\bfseries\upshape Solution]}
               {\end{proof}
                \renewcommand{\qedsymbol}{\oldqedsymbol}}
 
\begin{document}
 
\title{Symplectic Geometry}
\author{Regoon Wang  
\footnote{ChemE@UNSW, wang.regoon@gmail.com}} 


\maketitle
\begin{abstract}
This is a note while I reading Ana Cannas's \textit{Lectures on Symplectic Geometry}. However, it's not just a copy. 
It contains my understanding, questions and solutions to homework. I believe it's a good way for me to self-study 
mathematics. Make it slow and carefully, learn it by doing it.
\end{abstract} 

\tableofcontents
\section{Notations}
In order to keep the text short, common used notations are introduced here.
\begin{enumerate}
\item[$\bullet$] V be an \textit{m}-dimensional vector space over $\mathbb{R}$. 
\item[$\bullet$] $\Omega:V\times V\rightarrow \mathbb{R}$ be a bilinear map.
\end{enumerate}
\section{Symplectic Forms}
\begin{definition}
The map $\Omega$  is \tb{skew-symmetric} if $\Omega(u,v)=-\Omega(v,u), \forall u,v\in V$.
\end{definition}
\begin{thm}[\tb{Standard Form for Skew-symmetric Bilinear Map}]
$\exists$ a basis $u_1,\cdots,u_k,e_1,\cdots,e_n,f_1,\cdots,f_n$ of V s.t. $\forall i,j \text{ and } v\in V$
\begin{equation*}
\Omega(u_i,v)=\Omega(e_i,e_j)=\Omega(f_i,f_j)=0,\quad\Omega(e_i,f_j)=\delta_{ij}
\end{equation*}
\end{thm}

\tb{Remark.} 
\begin{enumerate}
\item[1.] The basis is not unique, though it is traditionally also called a "canonical" basis.
\item[2.] In matrix notation with respect to such basis, we have 
\begin{equation*}
\Omega(u,v)=
\begin{bmatrix}
- & u & -
\end{bmatrix}
\begin{bmatrix}
0 & 0 & 0 \\
0 & 0 & I_n \\
0 &-I_n& 0
\end{bmatrix}
\begin{bmatrix}
| \\
v \\
|
\end{bmatrix}
\end{equation*}
and all normed linear independt enginvectors are basis. 
\end{enumerate}

\begin{proof}
This induction proof is a skew-symmetric version of the Gram-Schmidt process. \\ Let $U:=\{u\in V\mid 
\Omega(u,v)=0,\forall v\in V\}$. Choose a basis $u_1,\cdots,u_k$, and choose a complemetary space W,
$$V=U\oplus W$$. Take any nonzero $e_1\in W$. Then there is $f_1\in W$ s.t. $\Omega(e_1,f_1)=1$. Let
\begin{align*}
W_1 &= span\{e_1,f_1\} \\
W_1^{\Omega}&=\{w\in W\mid \Omega(w,v)=0,\forall v\in W_1\}
\end{align*}
\begin{claim}
$W_1\cap W_1^{\Omega}={0}$.
\end{claim}
Suppose that $v=ae_1+bf_1\in W_1\cap W_1^{\Omega}$.
\begin{equation*}
 \left.\begin{aligned}
        0&=\Omega(v,e_1)=-b\\
        0&=\Omega(v,f_1)=a
       \end{aligned}
 \right\}
 \Rightarrow v=0
\end{equation*}
\begin{claim}
$W=W_1\oplus W_1^\Omega$.
\end{claim}
Suppose that $v\in W$ has $\Omega(v,e_1)=c, \Omega(v,f_1)=d$. Then
\[
 v = \underbrace{(-cf_1+de_1)}_{\in W_1}
   + \underbrace{(v+cf_1-de_1)}_{\in W_1^\Omega}
\]
Go on with $W_1^\Omega$: choose $e_2,f_2\in W_1^\Omega$ s.t. $\Omega(e_2,f_2)=1$, let $W_2=span{e_2,f_2}$, etc.
This process eventually stops because $\text{dim}V<\infty$. We hence obtain $$V=U\oplus W_1\oplus \cdots \oplus W_n$$.
\tb{Remark.}
\begin{enumerate}
\item $k:=\text{dim}U$ is an invariant of $(V,\Omega)$.\\
\item n is an invariant of $(V,\Omega)$; 2n is called the \tb{rank} of $\Omega$.
\end{enumerate}
\end{proof}
\subsection{Skew-Symmetric Bilinear Maps}
\begin{definition}
The map $\tilde{\Omega}:V\rightarrow V^*$ is the linear map defined by $\tilde{\Omega}(v)(u)=\Omega(v,u)$.
\end{definition}
\subsection{Symplectic Vector Space}
\subsection{Symplectic Manifolds}
\subsection{Symplectomorphisms}
\subsection{Homework}
\section{Symplectic Form on the Cotangent Bundle}
\subsection{Cotangent Bundle}
\subsection{Tautological and Canonical Forms in Coordinates}
\subsection{Coordinate-Free Definitions}
\subsection{Naturality of the Tautological and Canonical Forms}
\subsection{Homework} 
\section{Lagrangian Submanifolds}
\subsection{Submanifolds}
\subsection{Lagrangian Submanifolds of $ T^*X $}
\subsection{Conormal Bundles}
\subsection{Application to Symplectomorphisms}
\subsection{Homework}
\section{Generating Functions}
\subsection{Constructing Symlectomorphisms}
\subsection{Method of Generating Functions}
\subsection{Application to Geodesic Flow}
\subsection{Homework}
\section{Recurrence}
\subsection{Periodic Points}
\subsection{Billiards}
\subsection{Poincar\'{e} Recurrence}
\subsection{Homework}
\section{Preparation for the Local Theory}
\subsection{Isotopies and Vector Fields}
\subsection{Tubular neighborhood Theorem}
\subsection{Homotopy Formula}
\subsection{Homework}
\section{Moser Theorems}
\subsection{Notions of Equivalence for Symplectic Structure}
\subsection{Moser Trick}
\subsection{Moser Local Theorem}
\subsection{Homework}
\section{Darboux-Moser-Weinstein Theory}
\subsection{Classical Darboux Theorem}
\subsection{Lagrangian Subspaces}
\subsection{Weinstein Lagrangian Neighborhood Theorem}
\subsection{Homework}

\end{document}