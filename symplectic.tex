\documentclass[4paper]{article}

\usepackage[margin=1in]{geometry} 
\usepackage{amsmath,amsthm,amssymb}
\usepackage{textcomp}
\usepackage[colorlinks,linkcolor=red,anchorcolor=blue,citecolor=green]{hyperref}

\newcommand{\N}{\mathbb{N}}
\newcommand{\Z}{\mathbb{Z}}
\newcommand{\R}[1]{\mathbb{R}^{#1}}
\newcommand{\tb}[1]{\textbf{#1}}
\newcommand{\ti}[1]{\textit{#1}}
\newcommand{\mca}[1]{\mathcal{#1}}
\newcommand{\map}[3]{{#1}:{#2}\rightarrow{#3}}

\newtheorem{problem}{Problem}
\newtheorem{thm}{Thm.}[section]
\newtheorem{prop}{Prop.}[section]
\newtheorem{lem}{Lem.}[section]
\newtheorem{cor}{Coro.}[section]
%\newtheorem{conj}[thm]{Conjecture}
\newtheorem{exer}{Exer}[section]

\theoremstyle{definition}
\newtheorem{definition}{Def.}[section]
%\newtheorem{example}{Example}[section] 

\theoremstyle{remark}
\newtheorem*{note}{Note}
\newtheorem*{remark}{\tb{Remark}}
\newtheorem*{claim}{\tb{Claim}}
\newenvironment{solution}
               {\let\oldqedsymbol=\qedsymbol
                \renewcommand{\qedsymbol}{$\blacktriangleleft$}
                \begin{proof}[\bfseries\upshape Solution]}
               {\end{proof}
                \renewcommand{\qedsymbol}{\oldqedsymbol}}
 
\begin{document}
 
\title{Symplectic Geometry}
\author{Regoon Wang  
\footnote{ChemE@UNSW, wang.regoon@gmail.com}} 


\maketitle
\begin{abstract}
This is a note while I reading Ana Cannas's \textit{Lectures on Symplectic Geometry}. However, it's not
just a copy. It contains my understanding, questions and solutions to homework. I believe it's a good 
way for me to self-study mathematics. Make it slow and carefully, learn it by doing it.
\end{abstract} 

\tableofcontents
\section{Notations}
In order to keep the text short, common used notations are introduced here.
\begin{itemize}
\item V be an \textit{m}-dimensional vector space over $\mathbb{R}$. 
\item $\Omega:V\times V\rightarrow \mathbb{R}$ be a bilinear map.
\item $(V,\Omega)$ is a \tb{symplectic vector space}.
\end{itemize}
\section{Symplectic Forms}
\begin{definition}
The map $\Omega$  is \tb{skew-symmetric} if $\Omega(u,v)=-\Omega(v,u), \forall u,v\in V$.
\end{definition}
\begin{thm}[\tb{Standard Form for Skew-symmetric Bilinear Map}]\label{thm:2.1}
$\exists$ a basis $u_1,\cdots,u_k,e_1,\cdots,e_n,f_1,\cdots,f_n$ of V s.t. $\forall i,j \mathrm{ and } 
v\in V$
\begin{equation*}
\Omega(u_i,v)=\Omega(e_i,e_j)=\Omega(f_i,f_j)=0,\quad\Omega(e_i,f_j)=\delta_{ij}
\end{equation*}
\end{thm}

\tb{Remark.} 
\begin{enumerate}
\item[1.] The basis is not unique, though it is traditionally also called a "canonical" basis.
\item[2.] In matrix notation with respect to such basis, we have 
\begin{equation*}
\Omega(u,v)=
\begin{bmatrix}
- & u & -
\end{bmatrix}
\begin{bmatrix}
0 & 0 & 0 \\
0 & 0 & I_n \\
0 &-I_n& 0
\end{bmatrix}
\begin{bmatrix}
| \\
v \\
|
\end{bmatrix}
\end{equation*}
and all normed linear independt enginvectors are basis. 
\end{enumerate}

\begin{proof}
This induction proof is a skew-symmetric version of the Gram-Schmidt process. \\ Let $U:=\{u\in V\mid 
\Omega(u,v)=0,\forall v\in V\}$. Choose a basis $u_1,\cdots,u_k$, and choose a complemetary space W,
$$V=U\oplus W$$. Take any nonzero $e_1\in W$. Then there is $f_1\in W$ s.t. $\Omega(e_1,f_1)=1$. Let
\begin{align*}
W_1 &= span\{e_1,f_1\} \\
W_1^{\Omega}&=\{w\in W\mid \Omega(w,v)=0,\forall v\in W_1\}
\end{align*}
\begin{claim}
$W_1\cap W_1^{\Omega}={0}$.
\end{claim}
Suppose that $v=ae_1+bf_1\in W_1\cap W_1^{\Omega}$.
\begin{equation*}
 \left.\begin{aligned}
        0&=\Omega(v,e_1)=-b\\
        0&=\Omega(v,f_1)=a
       \end{aligned}
 \right\}
 \Rightarrow v=0
\end{equation*}
\begin{claim}
$W=W_1\oplus W_1^\Omega$.
\end{claim}
Suppose that $v\in W$ has $\Omega(v,e_1)=c, \Omega(v,f_1)=d$. Then
\[
 v = \underbrace{(-cf_1+de_1)}_{\in W_1}
   + \underbrace{(v+cf_1-de_1)}_{\in W_1^\Omega}
\]
Go on with $W_1^\Omega$: choose $e_2,f_2\in W_1^\Omega$ s.t. $\Omega(e_2,f_2)=1$, let $W_2=span{e_2,
f_2}$,etc. This process eventually stops because $\mathrm{dim}V<\infty$. We hence obtain $$V=U\oplus
W_1\oplus \cdots \oplus W_n$$.
\tb{Remark.}
\begin{enumerate}
\item $k:=\mathrm{dim}U$ is an invariant of $(V,\Omega)$.\\
\item n is an invariant of $(V,\Omega)$; 2n is called the \tb{rank} of $\Omega$.
\end{enumerate}
\end{proof}
\subsection{Skew-Symmetric Bilinear Maps}
\begin{definition}
The map $\tilde{\Omega}:V\rightarrow V^*$ is the linear map defined by $\tilde{\Omega}(v)(u)=\Omega
(v,u)$.
\end{definition}
\begin{definition}
A skew-symetric bilinear map $\Omega$ is \tb{symplectic} (or nondegenerate) if $\tilde{\Omega}$ is 
bijective, i.e., $U={0}$. The map $\Omega$ is then called a \tb{linear symplectic structure} on V,
and $(V,\Omega)$ is called a \tb{symplectic vector space}.
\end{definition}
\begin{note} 
These are immediate properties of symplectic map:
\begin{enumerate}
\item Duality: the map $\Omega : V \overset{\backsimeq}{\rightarrow} V^*$ is a bijection.
\item $\mathrm{dim}U=0,\mathrm{dim}V=2n$.
\item $(V,\Omega)$ has a basis $e_1,\cdots,e_n,f_1,\cdots,f_n$ s.t.
\begin{equation*}
\Omega(u,v)=
\begin{bmatrix}
- & u & -
\end{bmatrix}
\begin{bmatrix}
0    & I_n \\
-I_n & 0
\end{bmatrix}
\begin{bmatrix}
| \\
v \\
|
\end{bmatrix}
\end{equation*}
\end{enumerate}
\end{note}
\begin{remark}
Not all subspace W of a $(V,\Omega)$ look the same:
\begin{itemize}
\item W is \tb{sysmplecitc} if $\Omega\mid_W$ is nondegenerate, for instance $W=span{e_1,f_1}$.
\item W is \tb{isotropic} if $\Omega\mid_W\equiv0$, for instance $W=span{e_1,e_1}$.
\end{itemize}
\end{remark}
\subsection{Symplectic Vector Space}
\begin{definition}
A \tb{symplectomorphism} $\varphi$ between $(V,\Omega)$ and $(V',\Omega ')$ is a linear isomorphism
$\varphi:V\overset{\backsimeq}{\rightarrow} V'$ s.t. $\varphi^*\Omega '=\Omega, (\varphi^*
\Omega ')(u,v)=\Omega '(\varphi(u),\varphi(v))$. If a symplectomorphism exists, these two spaces are
said to be \tb{symplectomophic}.
\end{definition}
\begin{remark}
Thm \ref{thm:2.1} shows that any symplectic space is symplectomorphic to $(\R{2n},\Omega_0)$.
\end{remark}
\subsection{Symplectic Manifolds}
Let $\omega$ be a de Rham 2-form on a manifold M, that is, for each $p\in M$, the map $\omega_p:T_pM
\times T_pM\rightarrow \mathbb{R}$ is skew-symmetric bilinear on the tangent space to M at p, and
$\omega_p$ varies smoothly in p. We say that $\omega$ is closed if it satisfies the differential
equation $d\omega=0$, where d is the de Rham differential.
\begin{definition}
The 2-form $\omega$ is symplectic if $\omega$ is closed and $\omega_p$ is symplectic for all $p\in M$.
\end{definition}
If $\omega$ is symplectic, then dim$T_mM$=dim$M$ must be even.
\begin{definition}
A symplectic manifold is a pair $(M,\omega)$ where M is a manifold and $\omega$ is a symplectic form.
\end{definition}
\begin{example}
Let $M=\R{2n}$ with linear coordinates $x_1,\cdots,x_n,y_1,\cdots,y_n$. Then the form $$\omega_0 =
\sum_{i=1}^{n}dx_i\wedge dy_i$$ is symplectic, and the set $$\left\{\left(\frac{\partial}{\partial x_1}\right),
\cdots,\left(\frac{\partial}{\partial x_n}\right),\left(\frac{\partial}{\partial y_1}\right),\cdots,
\left(\frac{\partial}{\partial y_n}\right)\right\}$$ is a symplectic basis.
\end{example}
\begin{example}
Let $M=\mathbb{C}^n$ with linear coordinates $z_1,\cdots,z_n$. The form $$\omega_0=\frac{i}{2}\sum_{i=1}^{n}
dz_k\wedge d\overline{z}_k$$ is symplectic. In fact $\mathbb{C}^n\equiv\R{2n},z_k=x_k+\tb{i}y_k$.
\end{example}
\subsection{Symplectomorphisms}
\begin{definition}
Let $(M_1,\omega_1)$ and $(M_2,\omega_2)$ be 2n-dim symplectic manifolds, and let $\map{g}{M_1}{M2}$
be a diffeomorphism. Then g is a \tb{symplectomorphism} if $g^*\omega_2=\omega_1$.
\end{definition}
\begin{thm}[Darboux]
Let $(M,\omega)$ be 2n-dim sysmplectic manifold and point $p\in M$. Then there is a coordiante chart
$(\mathcal{U},x_1,\cdots,x_n,y_1,\cdots,y_n)$ centered at p s.t. the symplectic form is $$\omega=
\sum_{i=1}^{n}dx_i\wedge dy_i $$
\end{thm}
More precisely, any symplectic manifold $(M^{2n},\omega)$ is locally symplectomorphic to $(\R{2},\omega_0)$.
\subsection{Homework}
Given a linear subspace Y of a symplectic vector space $(V,\Omega)$, its \tb{symplectic orthogonal} is 
defined as $$Y^\Omega:=\{v\in V\mid\Omega(v,u)=0, \forall u\in Y\}.$$
\begin{exer}
$\mathrm{dim}Y+\mathrm{dim}Y^\Omega=\mathrm{dim}V$.
\end{exer}
\begin{exer}
$(Y^\Omega)^\Omega=Y$.
\end{exer}
\begin{exer}
if Y,W are subspace, then $Y\subseteq W\Leftrightarrow W^\Omega \subseteq Y^\Omega$.
\end{exer}
\begin{exer}
Y is symplectic $\Leftrightarrow Y\cap Y^\Omega=\{0\} \Leftrightarrow V=Y\oplus Y^\Omega$.
\end{exer}
\begin{exer}
Y is isotropic $\Rightarrow \mathrm{dim}Y\leq \frac{1}{2}\mathrm{dim}V$.
\end{exer}
\begin{exer}
An isotropic Y is called \tb{Lagrangian} when $\mathrm{dim}Y= \frac{1}{2}\mathrm{dim}V$. Check that: Y is
lagrangian $\Leftrightarrow$ Y is isotropic and coisotropic $\Leftrightarrow$ $Y=Y^\Omega$. 
\end{exer}
\begin{exer}
We call Y coisotropic when $Y^\Omega\Subset Y$. Check that every codimension 1 subspace Y is coisotropic.
\end{exer}
\section{Symplectic Form on the Cotangent Bundle}

\subsection{Cotangent Bundle}

\subsection{Tautological and Canonical Forms in Coordinates}

\subsection{Coordinate-Free Definitions}

\subsection{Naturality of the Tautological and Canonical Forms}

\subsection{Homework} 

\section{Lagrangian Submanifolds}

\subsection{Submanifolds}

\subsection{Lagrangian Submanifolds of $ T^*X $}

\subsection{Conormal Bundles}

\subsection{Application to Symplectomorphisms}

\subsection{Homework}

\section{Generating Functions}

\subsection{Constructing Symlectomorphisms}

\subsection{Method of Generating Functions}

\subsection{Application to Geodesic Flow}

\subsection{Homework}

\section{Recurrence}

\subsection{Periodic Points}

\subsection{Billiards}

\subsection{Poincar\'{e} Recurrence}

\subsection{Homework}

\section{Preparation for the Local Theory}

\subsection{Isotopies and Vector Fields}

\subsection{Tubular neighborhood Theorem}

\subsection{Homotopy Formula}

\subsection{Homework}

\section{Moser Theorems}

\subsection{Notions of Equivalence for Symplectic Structure}

\subsection{Moser Trick}

\subsection{Moser Local Theorem}

\subsection{Homework}

\section{Darboux-Moser-Weinstein Theory}

\subsection{Classical Darboux Theorem}

\subsection{Lagrangian Subspaces}

\subsection{Weinstein Lagrangian Neighborhood Theorem}

\subsection{Homework}

\section{Weinstein Tubular Neighborhood Theorem}

\subsection{Observation from linear algebra}

\subsection{Tubular Neighborhoods}

\subsection{Tangent space to the group of symplectomorphisms}

\subsection{Fixed points of symplectomorphisms}

\subsection{Homework}

\section{Contact Forms}

\subsection{Contact Structure}

\subsection{Examples}

\subsection{First Properties}

\subsection{Homework}

\end{document}